\documentclass[12pt,]{article}
\usepackage{lmodern}
\usepackage{amssymb,amsmath}
\usepackage{ifxetex,ifluatex}
\usepackage{fixltx2e} % provides \textsubscript
\ifnum 0\ifxetex 1\fi\ifluatex 1\fi=0 % if pdftex
  \usepackage[T1]{fontenc}
  \usepackage[utf8]{inputenc}
\else % if luatex or xelatex
  \ifxetex
    \usepackage{mathspec}
    \usepackage{xltxtra,xunicode}
  \else
    \usepackage{fontspec}
  \fi
  \defaultfontfeatures{Mapping=tex-text,Scale=MatchLowercase}
  \newcommand{\euro}{€}
\fi
% use upquote if available, for straight quotes in verbatim environments
\IfFileExists{upquote.sty}{\usepackage{upquote}}{}
% use microtype if available
\IfFileExists{microtype.sty}{%
\usepackage{microtype}
\UseMicrotypeSet[protrusion]{basicmath} % disable protrusion for tt fonts
}{}
\usepackage[margin=1in]{geometry}
\ifxetex
  \usepackage[setpagesize=false, % page size defined by xetex
              unicode=false, % unicode breaks when used with xetex
              xetex]{hyperref}
\else
  \usepackage[unicode=true]{hyperref}
\fi
\hypersetup{breaklinks=true,
            bookmarks=true,
            pdfauthor={},
            pdftitle={},
            colorlinks=true,
            citecolor=blue,
            urlcolor=blue,
            linkcolor=magenta,
            pdfborder={0 0 0}}
\urlstyle{same}  % don't use monospace font for urls
\setlength{\parindent}{0pt}
\setlength{\parskip}{6pt plus 2pt minus 1pt}
\setlength{\emergencystretch}{3em}  % prevent overfull lines
\setcounter{secnumdepth}{0}

%%% Use protect on footnotes to avoid problems with footnotes in titles
\let\rmarkdownfootnote\footnote%
\def\footnote{\protect\rmarkdownfootnote}

%%% Change title format to be more compact
\usepackage{titling}
\setlength{\droptitle}{-2em}
  \title{}
  \pretitle{\vspace{\droptitle}}
  \posttitle{}
  \author{}
  \preauthor{}\postauthor{}
  \date{}
  \predate{}\postdate{}


\usepackage{lineno}
\linenumbers
\usepackage{setspace}
\doublespacing


\begin{document}

\maketitle


\section{Forecasting climate change impacts on plant population dynamics
at large spatial extents: a test case with sagebrush (\emph{Artemisia})
species}\label{forecasting-climate-change-impacts-on-plant-population-dynamics-at-large-spatial-extents-a-test-case-with-sagebrush-species}

\subsubsection[Andrew T. Tredennick, Mevin B. Hooten, Cameron L.
Aldridge, Colin G. Homer, David T. Iles, Andrew Kleinhesselink, Eric
LaMalfa, Rebecca Mann, and Peter B. Adler]{Andrew T.
Tredennick\footnote{Corresponding author:
  \href{mailto:atredenn@gmail.com}{\href{mailto:atredenn@gmail.com}{\nolinkurl{atredenn@gmail.com}}}},
Mevin B. Hooten, Cameron L. Aldridge, Colin G. Homer, David T. Iles,
Andrew Kleinhesselink, Eric LaMalfa, Rebecca Mann, and Peter B.
Adler}\label{andrew-t.-tredennickcorrauth-mevin-b.-hooten-cameron-l.-aldridge-colin-g.-homer-david-t.-iles-andrew-kleinhesselink-eric-lamalfa-rebecca-mann-and-peter-b.-adler}

\emph{Andrew T. Tredennick, Department of Wildland Resources and the
Ecology Center, Utah State University, Logan, UT}

\emph{Mevin B. Hooten, U.S. Geological Survey, Colorado Cooperative Fish
and Wildlife Research Unit; Department of Fish, Wildlife, and
Conservation Biology, Colorado State University; Department of
Statistics, Colorado State University; Graduate Degree Program in
Ecology, Colorado State University, Fort Collins, CO}

\emph{Cameron L. Aldridge, Department of Ecosystem Science and
Sustainability, Colorado State University; Natural Resource Ecology
Laboratory, Colorado State University; U.S. Geological Survey, Fort
Collins Science Center; Graduate Degree Program in Ecology, Colorado
State University, Fort Collins, CO}

\emph{Colin G. Homer, U.S. Geological Survey (USGS) Earth Resources
Observation and Science (EROS) Center, Sioux Falls, SD}

\emph{David T. Iles, Department of Wildland Resources and the Ecology
Center, Utah State University, Logan, UT}

\emph{Andrew Kleinhesselink, Department of Wildland Resources and the
Ecology Center, Utah State University, Logan, UT}

\emph{Eric LaMalfa, Department of Wildland Resources and the Ecology
Center, Utah State University, Logan, UT}

\emph{Rebecca Mann, Department of Wildland Resources and the Ecology
Center, Utah State University, Logan, UT}

\emph{Peter B. Adler, Department of Wildland Resources and the Ecology
Center, Utah State University, Logan, UT}

\subsection{Summary}\label{summary}

\begin{enumerate}
\item Global environmental changes, like climate change, tend to play out at the landscape and regional scales. Thus, it is reasonable that attempts to forecast the impacts of such change do so at similar spatial scales. For plant species, this has meant relying primarily on a single tool, species distribution models, that contain little information on population dynamics and states.

\item Plant population models have proven exceedingly useful for projecting population dynamics and states under altered climate regimes, but most models are parameterized using “local” scale data from, generally, 1 $\text{m}^{2}$ plots. These models, even if highly predictive, cannot capture the full spatial variability in population responses to climate and thus are difficult to extrapolate from the micro to meso scale.

\item Here we build on density-structured population modeling approaches and describe a population model based on remote sensing pixels -- a `pixel-based' population model. At its core the model is simply an individual-based population model where, instead of individuals, we focus on pixels as super individuals. To demonstrate the approach, we model the rate of transitions among discrete percent cover classes using a 27 year remotely sensed time series of sagebrush (Artemesia spp.) cover in southwestern Wyoming. We use climate covariates in vital rate regressions so that we can forecast future population states under projected climate change.

\item The pixel-based population model successfully recovers the the mean and spread of the observed data when we use observed climate covariates. 

\item Our approach...
\end{enumerate}

\emph{Keywords}: population model, forecasting, integral projection
model, etc.

\subsection{Introduction}\label{introduction}

Forecasting the impacts of climate change on plant populations and
communities is a central challenge of modern ecology. The challenge lies
in faithfully representing population dynamics at spatial and temporal
scales relevant to policy and management decisions. However, almost
every study of plant population dynamics relies on demographic
observations recorded at the meter to sub-meter scale. Local-scale
demographic data make building population projection models an easy task
(Ellner et al., 2006; Rees et al., 2009; Adler et al., 2012), but it is
very difficult to extrapolate small-scale studies to large spatial
extents with any certainty because the data likely only represent a
small subset of parameter space and environmental conditions. Also,
given the costliness (in time and money) of collecting detailed
demographic data, it is unlikely that ``enough'' local-scale data can be
collected to accurately project large-scale population dynamics.

The fundamental limitation of local-scale data is that they are
insufficient for estimating the full extent of variation in model
parameters possible at larger spatial and temporal extents (Freckleton
et al., 2011; Queenborough et al., 2011). It is possible to estimate the
variance of population model parameters related to spatial and temporal
variability, but the observations used to estimate parameter uncertainty
tend to be clustered in space or time. Thus, extrapolating beyond the
spatial or temporal resolution of the data used to parameterize models
can lead to biased and wrong forecasts of future population states.

Alternatively, large-scale trends in populations are easily detected
using widely available monitoring data, but such data are rarely used to
project future population states. Given the tradeoffs between the
spatial-scale of data and the inference that can be drawn from the data,
we suggest combining the best features of local-scale population
modeling and large-scale monitoring data. Specifically, we propose a new
approach for modeling population dynamics at large spatial extents based
on the theory and mechanics of individual-based models (IBMs) as applied
to remotely-sensed time series data.

The approach we describe in this paper flows naturally from recent work
on density-structured population models. Density-structured models rely
on modeling the transitions of a population among discrete states rather
than the traditional approach of modeling the transitions of
individuals. Relatively few ecologists have taken up the
density-structured mantle first proposed by Taylor, C. M. et al. (2004),
but recently Freckleton et al. (2011) and Queenborough et al. (2011)
have higlighted the potential value the approach.

\subsection{Materials and Methods}\label{materials-and-methods}

\subsubsection{Data: Remotely-sensed time
series}\label{data-remotely-sensed-time-series}

To show our pixel-based modeling approach we use a subset of a
remotely-sensed time series of sagebrush (\emph{Artemisia} spp.) shrub
percent cover in Wyoming (Homer et al., 2012) (Figure
{[}fig:Location-of-the{]}). Sagebrush percent cover was estimated using
a regression tree to relate ground reflectances retrieved by three
sources of optical imagery (QuickBird, Landsat, and AWiFS) to 1,780
field observations of sagebrush cover distributed across Wyoming. The
regression tree model was further validated using another 500 field
observations. For Wyoming sagebrush, the model achieved an \(R^{2}\) =
0.65. To hind-cast sagebrush cover the regression tree model was applied
to historical remote sensing images. This resulted in an annual time
series of sagebrush cover at 30 m resolution from 1984 to 2011. In this
dataset, values represent the percentage of a 30 \(\times\) 30 m pixel
covered by sagebrush. For our purposes we use two 5 \(\times\) 5 km
subsets (study areas 1 and 2), totaling 27,948 pixels each. Thus, each
full dataset contains 754,596 observations (27 years \(\times\) 27,948
pixels).

\subsubsection{Data: Climate covariates}\label{data-climate-covariates}

Our approach is based on modeling how changes in plant populations
relate to climate variables. We a priori narrowed our focus to climate
covariates we know are important for sagebrush and that could be easily
calculated from general circulation model future projections. In all
vital rate regressions we focus on the demographic processes driving
changes in percent sagebrush cover in year \emph{t}-1 to year \emph{t},
and how these processes are mediated by climate. The four climate
variables in our vital rate regressions are: (1) year \emph{t} fall
through summer precipitation, (2) year \emph{t}-1 fall through summer
precipitation, (3) year \emph{t} average spring temperature, and (4)
year \emph{t}-1 average spring temperature (Table
{[}tab:Definition-of-climate{]}). Climate data to parameterize vital
rate regressions (`observed' climate) came from the PRISM climate
database. Climate data were extracted for the centroid our study area.

Climate data used to inform model projections of future sagebrush cover
came from average outputs of all models used in the most recent IPCC
report, the Coupled Model Intercomparison Project 5 (CMIP5;
\url{http://cmip-pcmdi.llnl.gov/cmip5/}). We downloaded model estimates
of historic climate (1950-2000) and projected climate (2050-2100) for
the centroid of our study area. We wanted to keep the temporal
variability from observed climate consistent, so our aim was to adjust
observed climate according to projected changes. For temperature we
simply subtracted the average historic temperature from the average
projected temperature. For precipitation we calculated the proportional
change: we subtracted average projected precipitation from average
historic precipitation, and then divided by average historic
precipitation. We did this for three ``Representative Concentration
Pathways'': RCP 4.5, RCP 6.0, and RCP 8.5. RCPs are described here,
\url{http://tntcat.iiasa.ac.at/RcpDb/}. Basically they correspond to
stabilization of radiative forcing before 2100, after 2100, and
increasing greenhouse gas emissions, respectively. We implemented
projected temperature and precipitation changes in the model by
perturbing observed annual values by the values in Table \#.

\begin{table}[h]
\protect\caption{\label{tab:Changes-in-climate}Changes in climate variables from CMIP5
average projections.}
\begin{tabular}{llllll}
\hline
 & \multicolumn{2}{c}{Study Area 1} &  & \multicolumn{2}{c}{Study Area 2}\tabularnewline 
Emissions Scenario & $\Delta$ temperature & $\Delta$ precipitation &  & $\Delta$ temperature & $\Delta$ precipitation\tabularnewline
\hline
RCP 4.5 & 2.98\textdegree & 8.94\% &  & 3.04\textdegree & 7.42\%\tabularnewline
RCP 6.0 & 3.13\textdegree & 8.64\% &  & 3.23\textdegree & 6.72\%\tabularnewline
RCP 8.5 & 4.79\textdegree & 11.0\% &  & 4.90\textdegree & 8.81\%\tabularnewline
\hline
\end{tabular}
\end{table}

\subsubsection{Additive spatio-temporal model for sagebrush
cover}\label{additive-spatio-temporal-model-for-sagebrush-cover}

We use a descriptive model for sagebrush cover that includes additive
spatial and temporal effects. Since the spatial and temporal effects are
additive, we are not considering space-time interactions. Our model aims
to relate observed sagebrush cover, \(C_{s,t}\), at time \emph{t} and
spatial location \emph{s} to an underlying process \(\mu_{s,t}\) that is
defined over all spatial locations and observation times. For
statistical convenience, we consider discrete values of sagebrush
percent cover as counts, giving us the formulation

\begin{align}
[\textbf{C}] &= \text{Poisson}(\boldsymbol{\lambda}) \\
\boldsymbol{\lambda} &= \text{exp}(\boldsymbol{\mu}) 
\end{align}

which states that counts (integer percent cover) are Poisson distributed
with a log link function on abundance intensity
(\(\boldsymbol{\lambda}\)). We assume the underlying process,
\(\boldsymbol{\mu}\), is a function of climate covariates (\textbf{X}),
lagged sagebrush abundace (a temporal process), and a purely spatial
random effect (\(\boldsymbol{\eta}\)):

\begin{align}
\mu_{s,t} = \underbrace{\beta_{0} + \beta_{1}\mu_{s,t-1}}_\text{temporal} + \underbrace{\mathbf{X}_{t}\boldsymbol{\gamma}}_\text{climate} + \underbrace{\eta_{s}}_\text{spatial} + \underbrace{\varepsilon_{s,t}}_\text{error}
\end{align}

Fitting the spatial random effect (\(\eta\)) is computationally
demanding when the observations come from 10s of 1,000s locations. The
dimensionality is just too large. To overcome these computational
constraints we employed a dimension reduction strategy to reduce the
number of parameters that need to be estimated to account for spatial
variation. A common strategy is to express high dimensional effects,
like \(\boldsymbol{\eta}\), as the product of an expansion matrix,
\textbf{K}, and a smaller parameter vector, \(\boldsymbol{\alpha}\). We
can then define the spatial effect as

\begin{align}
\boldsymbol{\eta} = \textbf{K}\boldsymbol{\alpha}.
\end{align}

In this case, \(\boldsymbol{\alpha}\) is a \(m \times 1\) vector of
reduced spatial random effects, and \textbf{K} is a \(S \times m\)
matrix that maps the reduced effects to the full \emph{S}-dimensional
space, where \emph{S} is the total number of observed locations. Thus,
we are able to reduce the effective number of parameters from
\({S} \rightarrow m\).

Obviously, the last remaining obstacle is to parameterize \textbf{K}.
Here we use kernel convolution to interpolate the spatial random effect
between \emph{m} ``knots'' that are nonrandomly distributed across the
space of our study area. These means we are modeling spatial random
effects at the knot level, and we use \textbf{K} to interpolate those
effects between knots. We use a simple Gaussian kernel density to define
the distance-decay function around the knots (\textbf{w}), such that the
entries of \textbf{K} are

\begin{align}
K_{s,m} &= w_{s,m}/\sum_{s} w_{s,m}, \text{where} \\
w_{s,m} &= (2 \pi \sigma^{2})^{-1} \text{exp}(\frac{-d_{s,m}}{2 \phi}),
\end{align}

where \(d_{s,m}\) is the Euclidean distance between the centroid of
sample cell \emph{s} and the location of knot \emph{m}, and \(\sigma\)
is the kernel bandwidth. Based on residual spatial structure we found
after fitting a simple model without spatial effects, we set \(\phi\) to
about one-third of the range of spatial dependence evident in the
residual variograms -- 250 meters in this case. In the end, the simplest
way to understand our dimension reduction strategy is to consider
\(\alpha\)s as weights for each knot.

\subsubsection*{Computing and model
estimation}\label{computing-and-model-estimation}
\addcontentsline{toc}{subsubsection}{Computing and model estimation}

\end{document}
